%%%%%%%%%%%%%%%%%%%%%%%%%%%%%%%%%%%%%%%%%%%%%%%%%%%%%%%%%%%%%%%%%%%%%%%%
%%%%%%%%%%%%%%%%%%%%%% Simple LaTeX CV Template %%%%%%%%%%%%%%%%%%%%%%%%
%%%%%%%%%%%%%%%%%%%%%%%%%%%%%%%%%%%%%%%%%%%%%%%%%%%%%%%%%%%%%%%%%%%%%%%%

%%%%%%%%%%%%%%%%%%%%%%%%%%%%%%%%%%%%%%%%%%%%%%%%%%%%%%%%%%%%%%%%%%%%%%%%
%% NOTE: If you find that it says                                     %%
%%                                                                    %%
%%                           1 of ??                                  %%
%%                                                                    %%
%% at the bottom of your first page, this means that the AUX file     %%
%% was not available when you ran LaTeX on this source. Simply RERUN  %% 
%% LaTeX to get the ``??'' replaced with the number of the last page  %% 
%% of the document. The AUX file will be generated on the first run   %%
%% of LaTeX and used on the second run to fill in all of the          %%
%% references.                                                        %%
%%%%%%%%%%%%%%%%%%%%%%%%%%%%%%%%%%%%%%%%%%%%%%%%%%%%%%%%%%%%%%%%%%%%%%%%

%%%%%%%%%%%%%%%%%%%%%%%%%%%% Document Setup %%%%%%%%%%%%%%%%%%%%%%%%%%%%

% Don't like 10pt? Try 11pt or 12pt
\documentclass[10pt]{article}

% This is a helpful package that puts math inside length specifications
\usepackage{calc}

% Layout: Puts the section titles on left side of page
\reversemarginpar

%
%         PAPER SIZE, PAGE NUMBER, AND DOCUMENT LAYOUT NOTES:
%
% The next \usepackage line changes the layout for CV style section
% headings as marginal notes. It also sets up the paper size as either
% letter or A4. By default, letter was used. If A4 paper is desired,
% comment out the letterpaper lines and uncomment the a4paper lines.
%
% As you can see, the margin widths and section title widths can be
% easily adjusted.
%
% ALSO: Notice that the includefoot option can be commented OUT in order
% to put the PAGE NUMBER *IN* the bottom margin. This will make the
% effective text area larger.
%
% IF YOU WISH TO REMOVE THE ``of LASTPAGE'' next to each page number,
% see the note about the +LP and -LP lines below. Comment out the +LP
% and uncomment the -LP.
%
% IF YOU WISH TO REMOVE PAGE NUMBERS, be sure that the includefoot line
% is uncommented and ALSO uncomment the \pagestyle{empty} a few lines
% below.
%

%% Use these lines for letter-sized paper
\usepackage[paper=letterpaper,
            %includefoot, % Uncomment to put page number above margin
            marginparwidth=1.2in,     % Length of section titles
            marginparsep=.05in,       % Space between titles and text
            margin=1in,               % 1 inch margins
            includemp]{geometry}

%% Use these lines for A4-sized paper
%\usepackage[paper=a4paper,
%            %includefoot, % Uncomment to put page number above margin
%            marginparwidth=30.5mm,    % Length of section titles
%            marginparsep=1.5mm,       % Space between titles and text
%            margin=25mm,              % 25mm margins
%            includemp]{geometry}

%% More layout: Get rid of indenting throughout entire document
\setlength{\parindent}{0in}

%% This gives us fun enumeration environments. compactenum will be nice.
\usepackage{paralist}

%% Reference the last page in the page number
%
% NOTE: comment the +LP line and uncomment the -LP line to have page
%       numbers without the ``of ##'' last page reference)
%
% NOTE: uncomment the \pagestyle{empty} line to get rid of all page
%       numbers (make sure includefoot is commented out above)
%
\usepackage{fancyhdr,lastpage}
\pagestyle{fancy}
\pagestyle{empty}      % Uncomment this to get rid of page numbers
\fancyhf{}\renewcommand{\headrulewidth}{0pt}
\fancyfootoffset{\marginparsep+\marginparwidth}
\newlength{\footpageshift}
\setlength{\footpageshift}
          {0.5\textwidth+0.5\marginparsep+0.5\marginparwidth-2in}
\lfoot{\hspace{\footpageshift}%
       \parbox{4in}{\, \hfill %
                    \arabic{page} of \protect\pageref*{LastPage} % +LP
%                    \arabic{page}                               % -LP
                    \hfill \,}}

% Finally, give us PDF bookmarks
\usepackage{color,hyperref}
\definecolor{darkblue}{rgb}{0.0,0.0,0.3}
\hypersetup{colorlinks,breaklinks,
            linkcolor=darkblue,urlcolor=darkblue,
            anchorcolor=darkblue,citecolor=darkblue}

%%%%%%%%%%%%%%%%%%%%%%%% End Document Setup %%%%%%%%%%%%%%%%%%%%%%%%%%%%


%%%%%%%%%%%%%%%%%%%%%%%%%%% Helper Commands %%%%%%%%%%%%%%%%%%%%%%%%%%%%

% The title (name) with a horizontal rule under it
%
% Usage: \makeheading{name}
%
% Place at top of document. It should be the first thing.
\newcommand{\makeheading}[1]%
        {\hspace*{-\marginparsep minus \marginparwidth}%
         \begin{minipage}[t]{\textwidth+\marginparwidth+\marginparsep}%
                {\large \bfseries #1}\\[-0.15\baselineskip]%
                 \rule{\columnwidth}{1pt}%
         \end{minipage}}

% The section headings
%
% Usage: \section{section name}
%
% Follow this section IMMEDIATELY with the first line of the section
% text. Do not put whitespace in between. That is, do this:
%
%       \section{My Information}
%       Here is my information.
%
% and NOT this:
%
%       \section{My Information}
%
%       Here is my information.
%
% Otherwise the top of the section header will not line up with the top
% of the section. Of course, using a single comment character (%) on
% empty lines allows for the function of the first example with the
% readability of the second example.
\renewcommand{\section}[2]%
        {\pagebreak[2]\vspace{1.3\baselineskip}%
         \phantomsection\addcontentsline{toc}{section}{#1}%
         \hspace{0in}%
         \marginpar{
         \raggedright \scshape #1}#2}

% An itemize-style list with lots of space between items
\newenvironment{outerlist}[1][\enskip\textbullet]%
        {\begin{enumerate}[#1]}{\end{enumerate}%
         \vspace{-.6\baselineskip}}

% An itemize-style list with little space between items
\newenvironment{innerlist}[1][\enskip\textbullet]%
        {\begin{compactenum}[#1]}{\end{compactenum}}

% To add some paragraph space between lines.
% This also tells LaTeX to preferably break a page on one of these gaps
% if there is a needed pagebreak nearby.
\newcommand{\blankline}{\quad\pagebreak[2]}

%%%%%%%%%%%%%%%%%%%%%%%% End Helper Commands %%%%%%%%%%%%%%%%%%%%%%%%%%%

%%%%%%%%%%%%%%%%%%%%%%%%% Begin CV Document %%%%%%%%%%%%%%%%%%%%%%%%%%%%

\begin{document}
\makeheading{Ilya Yakubovich}

\section{Contact Information}
%
% NOTE: Mind where the & separators and \\ breaks are in the following
%       table.
%
% ALSO: \rcollength is the width of the right column of the table 
%       (adjust it to your liking; default is 1.85in).
%
\newlength{\rcollength}\setlength{\rcollength}{3.0in}%
%
\begin{tabular}[t]{@{}p{\textwidth-\rcollength}p{\rcollength}}
2600 Bryant Street, Apt 2     & \textit{Phone:} (415) 690-7342 \\
San Francisco, CA             & \textit{Portfolio:} \href{http://yak.nu}{http://yak.nu}  \\
USA 94110       & \textit{E-mail:}
\href{mailto:ilya.yakubovich@gmail.com}{ilya.yakubovich@gmail.com}\\
\end{tabular}

\section{Summary}
I'm a Human/Computer Interaction designer with a computer science background
and over 6 years of experience in development, product management, and user 
experience design. Please view my portfolio for examples of work, and an outline
of my design process.

\section{Experience}
{\textbf{  \href{http://www.yammer.com}{Yammer, Inc.}  }}
\begin{outerlist}
\item[] \textit{Product Manager / Lead User Experience Designer}%
        \hfill \textbf{2008 --- 2011}
\begin{innerlist}
\item Yammer is an enterprise social networking solution that connects employee communities within large
      organizations.
\item As a member of the founding team, I designed a product that is now used at over 100,000
      businesses worldwide.
\item Created mockups (Omnigraffle) and high-fidelity comps (Photoshop).
\item Wrote feature specifications and design documents.
\item Implemented user-facing features in HTML, CSS, and Ruby on Rails.
\item Worked with developers during the implementation phase.
\item Set up and maintained a ticketing system (JIRA).
\item Set up and analyzed A/B tests to measure the impact of new features.
\item Participated in the Quality Assurance process. \\
\end{innerlist}
\end{outerlist}


{\textbf{ \href{http://www.geni.com}{Geni, Inc.}  }}
\begin{outerlist}
\item[] \textit{Product Manager}%
        \hfill \textbf{2007 --- 2008}
\begin{innerlist}
\item Geni is a collaborative genealogy tool that is used by over 6 million users.
\item Brainstormed new feature ideas with the product team.
\item Created mockups and high-fidelity comps.
\item Wrote feature specifications and design documents.
\item Worked with developers during the implementation phase.\\
\end{innerlist}
\end{outerlist}


{\textbf{Device Scientific Inc.}}
\begin{outerlist}

\item[] \textit{Software developer}%
        \hfill \textbf{2006}
\begin{innerlist}
\item Implemented Linux device drivers for new linescan camera device (C++, ARM Linux).
\item Interfaced Linux hardware and a .NET application on a host machine.
\item Wrote a testing and debugging harness for hardware, and assisted in the hardware debugging process.
\item Worked in a distributed team of 5 developers.\\
\end{innerlist}
\end{outerlist}


\textbf{NSIX Vision Inc.}
\begin{outerlist}
\item[] \textit{Designer / Developer}%
        \hfill \textbf{2006}
\begin{innerlist}
\item Designed and implemented a website for a consulting startup.
\item Developed dynamic web-pages (HTML, CSS, ASP, SQL server).
\item Designed logos and slideshow presentations.\\
\end {innerlist}
\end{outerlist}

\pagebreak

{\textbf{York University Computer Club }}
\begin{outerlist}
\item[] \textit{President}%
        \hfill \textbf{2006 --- 2008}
\begin{innerlist}
\item Set up and maintained Linux, BSD and Solaris servers and workstations.
\item Tutored first and second year CS students.
\item Created educational materials for first year CS students.\\
\end{innerlist}
\end{outerlist}



{\textbf{Irina Guletsky, PhD}}
\begin{outerlist}
\item[] \textit{Software developer}%
        \hfill \textbf{2005}
\begin{innerlist}
\item Designed and implemented C\# application used as a tool for music theory research.
\item The application involved data entry, data analysis and graphical output (representation of researched musical patterns).\\
\end {innerlist}
\end{outerlist} 



\section{Education}
  \href{http://www.cs.yorku.ca/}{\textbf{York University}}
  Toronto, Ontario

  \begin{innerlist}
    \item[] BSc, Computer Science, 2008 \\
  \end{innerlist}


  \href{http://www.thornhill.ss.yrdsb.edu.on.ca}
       {\textbf{Thornhill Secondary Secondary (TSS)}}
       Toronto, Ontario
  \begin{innerlist}
    \item[] OSSD, June 2004 
  \end{innerlist} 




\section{Skills} 
  \textbf{Tools:}                          Omnigraffle, Photoshop, Illustrator, VIM.                          \\
  \textbf{Programming Languages:}          Ruby on Rails, C, Java, HTML, CSS. Familiar with Javascript.       \\
  \textbf{Operating Systems:}              Developed for Microsoft Windows and Linux.                         \\
  \textbf{Languages:}                      Fluent in English, Hebrew and Russian.                             \\
\end{document}
